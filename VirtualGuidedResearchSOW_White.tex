\documentclass{article}

\usepackage{amsmath}
\usepackage{amssymb}
\usepackage{graphicx}
\usepackage{fullpage}
\usepackage{enumerate}
\usepackage{color}
\usepackage[inline,shortlabels]{enumitem}
\usepackage{soul}
\title{}
\parindent=0in
\pagenumbering{gobble}
\usepackage{multirow}
\usepackage[margin=1in]{geometry}

\title{FIT 2020 Guided Research Apprenticeship}
\date{}

\begin{document}

\maketitle
\vspace{-1.5cm}
\subsection*{Project: Predicting the Spread of Infectious Disease}

\bigskip

\textbf{Description}: The rapid spread of the coronavirus pandemic throughout the world underscores the importance of tracking and predicting the spread of infectious disease in order to learn how to minimize its extent and mobilize treatment resources in an effective way. The participants will be introduced to some practical aspects of calculus, learn to write code in the in-demand Python language, and use some high-speed mathematical algorithms from the powerful NumPy and SciPy libraries. The participants will use real-world data on the spread of the current pandemic to learn about models in epidemiology that make these predictions and complete a project where they will each construct their own novel model, determine optimal parameters to fit the model to the data with code, and explore the effectiveness of different responses. Projects may serve as a foundation for a capstone project or publication beyond the Virtual Summer Apprenticeship Program, and outstanding student methods may appear in a web app currently in development.

\bigskip

Four student reseachers will have live instruction and activities with Dr. White for \textbf{a minimum of 36 hours} along with activities to complete outside the meetings over a span of \textbf{5 weeks (10 June to 15 July)}. By the end of the program, researchers will produce:
\begin{enumerate}
\item Python programs for simulating epidemics, model-fitting, and forecasting under various conditions
\item A final report on methods and findings submitted to Dr. White \textbf{(15 July)}
\item A live webcasted presentation of their findings in a live webcast open to family, friends, or anyone the students wish to invite \textbf{(17 July)}
\end{enumerate}

\subsection*{Tentative Schedule}

\begin{itemize}
\item \textbf{Pre-Program}: Poll students on their knowledge of programming and mathematics to determine precisely which details need to be taught, determine if the students refer to work alone on their final project or in some groups, and schedule weekly times for live meetings.

\item \textbf{Week 1}: Introduction, guide students to register for Google Colab for cloud-based collaborative Python notebooks and a Discord server for discussion boards, instant messaging, and voice chat.

Instruction on limits and derivatives, epidemiological models, integrals and mathematical analysis of compartment models, programming in Python, simulating models in Python \textbf{(8 hours)}

\item \textbf{Week 2}: Instruction on measuring error between models and real data, brute-force optimization, 1D optimization by hand and by gradient descent algorithms, and retrieving real-world data from Johns Hopkins or other sources with Python \textbf{(8 hours)}

\item \textbf{Week 3}: Instruction on partial derivatives in two variables, 2D optimization by hand, gradient descent in 2D to minimize error in model-fitting, and extension to any finite number of variables \textbf{(8 hours)}

\item \textbf{Weeks 4-5}: Previous weeks provide researchers with all the necessary tools for modern modeling, so we next focus on simulating epidemics under mitigation strategies (vaccination, social distancing, travel restrictions, and others). Researchers will compare strategies by implementing the strategies in their models, simulating, and comparing forecasts. \textbf{(12 hours)}

\bigskip

Research groups will also use this time to produce final reports written in the form of an academic research paper, with background, a description of methods, results and main findings, and conclusions.

\item \textbf{Post-Program}: The virtual presentation will take place on July 17.

\subsection*{Potential Further Work}

Outstanding new methods or especially interesting results may be suitable for further work in several possible directions:

\begin{itemize}
\item Students can use newly obtained skills to extend their research independently in epidemiology or mathematical modeling of other phenomena.

\item Methods may be incorporated in a web app under development by Dr. White's FIT students for modeling pandemics and educational purposes. Researchers will be credited by name online.

\item Dr. White is willing to supervise and collaborate for further research and publication on epidemiology if results show promise to create significant new knowledge.
\end{itemize}

\end{itemize}

\end{document}
